% Exposition on Emotiq Crypto
% DM/Emotiq  02/18
%
\documentclass[article,oneside]{memoir}
\usepackage{geometry} 
\geometry{letterpaper} 
%\usepackage[parfill]{parskip}    % Activate to begin paragraphs with an empty line
\usepackage{graphicx}
\usepackage{amsmath}
\usepackage{amssymb}
\usepackage{epstopdf}
\usepackage[latin1]{inputenc}
\usepackage{fancyhdr}
\usepackage{dcolumn}
\usepackage[pdftex,bookmarks]{hyperref}
\usepackage{moreverb}
\usepackage{listings}

\DeclareGraphicsRule{.tif}{png}{.png}{`convert #1 `dirname #1`/`basename #1 .tif`.png}

% \usepackage[draft]{pdfdraftcopy}
% \draftstring{PRELIMINARY}
% \definecolor{my-draft-color}{rgb}{0.85,0.85,0.85}
% \draftcolor{my-draft-color}
%      \watermarkgraphic{/usr/local/lib/Logo75Img-Alpha25y.pdf}
% \watermark


%% -----------------------------------------------------------------------------------------------------------------------
\pagestyle{fancy}
\fancyhf{}  %delete current header and footer
\fancyhead[L]{\bfseries{Emotiq Crypto Features}}
\fancyhead[R]{\bfseries\thepage}
\fancyfoot[L]{\small{Copyright {\copyright} 2018 by Emotiq AG}}
\renewcommand{\headrulewidth}{0.5pt}
\renewcommand{\footrulewidth}{0.5pt}
\addtolength{\headheight}{0.5pt} % space for the rule

%\fancypagestyle{plain}{%
%	\fancyhead{} % get rid of headers on plain pages
%	\renewcommand{\headrulewidth}{0pt} % and the line
%	}
	
%% -----------------------------------------------------------------------------------------------------------------------
\title{Emotiq Crypto Features}
\author{David McClain, Emotiq AG\\dbm@emotiq.ch\\1st Draft}
%\date{\today}  % leave commented out for current date to show
%\date{Friday, June 15, 2007}

\begin{document}
\maketitle

\tableofcontents*

%% -----------------------------------------------------------------------------------------------------------------------
%% Our Commentary

%%%%%%%%%%%%%%%%%%%%%%%%%%%%%%%%%%%%%%%%%%%%%%%
\lstset{ %
language=Lisp,                % choose the language of the code
basicstyle=\footnotesize,       % the size of the fonts that are used for the code
numbers=left,                   % where to put the line-numbers
numberstyle=\footnotesize,      % the size of the fonts that are used for the line-numbers
stepnumber=2,                   % the step between two line-numbers. If it's 1 each line will be numbered
numbersep=5pt,                  % how far the line-numbers are from the code
backgroundcolor=\color{white},  % choose the background color. You must add \usepackage{color}
showspaces=false,               % show spaces adding particular underscores
showstringspaces=false,         % underline spaces within strings
showtabs=false,                 % show tabs within strings adding particular underscores
frame=single,	                % adds a frame around the code
tabsize=2,	                % sets default tabsize to 2 spaces
captionpos=b,                   % sets the caption-position to bottom
breaklines=true,                % sets automatic line breaking
breakatwhitespace=false,        % sets if automatic breaks should only happen at whitespace
escapeinside={\%*}{*)}          % if you want to add a comment within your code
}

%%%%%%%%%%%%%%%%%%%%%%%%%%%%%%%%%%%%%%%%%%%%%%%

\chapter{Pairing-based Cryptography}

Emotiq utilizes advanced bilinear pairing-based cryptography\cite{thesis}\cite{lib} (PBC) for user keying, Boneh-Lynn-Shacham (BLS) Signatures\cite{bls}, fast multi-party signatures, and for Randomness Generation. The advantages of PBC are numerous and include short signatures, fast signature generation, safe deterministic hierarchical wallet keying, and fast multiparty randomness generation.

A bilinear pairing uses pairs of Elliptic Curves, defined over two separate groups, such that their bilinear mappings produce homomorphic encryption in a resulting composite field. If we denote the two curve groups as $G1$ and $G2$,  their pairing field $GT$, and prime order finite field $Zr$, then their pairing $e(G1,G2) \in GT$ is such that $$e(a U, V) = e(U, a V)= g^a$$ where $U \in G1$, $V \in G2$, $a \in Zr$, and $g \in GT$.

In our system group $G1$ is always the smaller group, with the shortest representation. Specifically, our $Zr$ uses 256 bits, $G1$ was chosen to have a 264-bit representation, $G2$ has a 520-bit representation, $GT$ has a 3072-bit representation, and the prime order of the groups is $q \approx  2^{254}$, which gives us roughly $2^{127}$ security.

Private keys belong to the finite field $Zr$ with the same prime group order. Public keys are generated in $G2$, and signatures are generated in $G1$. The embedding degree of our curves is 12, and correspond to {\emph{Type f}} asymmetric pairing curves in Lynn's Thesis\cite{thesis}. Wherever they occur, we use  compressed point representation for group elements from $G1$ and $G2$.

\chapter{Boneh-Lynn-Shacham (BLS) Signatures}

BLS signatures are the shortest possible, and enable multisignature generation in just one pass. A BLS signature on message $msg$ is computed as $$sig = s G1(H(msg))$$ where $H(x)$ is the SHA3/256 hash of its argument, $s \in Zr$ is the user's secret key value, and $G1(H(x)) \in G1$ is the group member that corresponds to that hash value. A signature is always accompanied by the public key of the signer, $P = s V$, for generator $V \in G2$,  producing a signed message as a triple $$(msg, sig, P)$$

Because of homomorphism we can verify a signature by noting that a valid signature exhibits the pairing relationship $$e(s  G1(H(msg)),V) = e(G1(H(msg)), s V) = e(G1(H(msg)), P)$$

And also because of homomorphism, we can easily compute a multi-party signature by simply summing the individual signatures and also summing their corresponding public keys: $$e({\sum_i s_i} G1(H(msg)),V) = e(G1(H(msg)), {\sum_i P_i})$$ producing the collective triple $$(msg, {\sum_i sig_i}, {\sum_i P_i})$$

Therefore, during the computation of collective signatures, we need only a single pass through all participants as we gather and sum their signture components. A collective signature appears no different than a single signature.

In contrast, conventional Schnorr signatures require two signature values, forming a quadruple with message and public key. For message $msg$ the Schnorr signature is the pair $(R,u)$ of an Elliptic Curve point $R$ and a field value $u$, where $R = r G$, for generator point $G$, and $r = H(k_{rand}, msg, P)$ is chosen as a random offset. Finally $u = r + H(R,P,msg) s$. The Schnorr signature is validated by checking that $$u G = R + H(R, P, msg) P$$

For collective Schnorr signing, all participants are asked to compute their own commitments $R_i = r_i G$. Those values are collected and summed to produce a global challenge value, $c_{glb} = H(\sum_i R_i, \sum_i P_i, msg)$.  Then the participants are asked to produce their $u_i$ values against that global challenge:

$$u_i = r_i + c_{glb} s$$

and again the values are summed. Hence collective Schnorr signatures require two interactions with every signer of the message. Network traffic is approximately twice that required for BLS signatures, with a consequent window of opportunity for attackers to spoil the process during the second round.

\chapter{Fast Randomness Generation with PVSS}

The use of BLS Signatures allows an abbreviated form of PVSS randomness generation. Participants in randomness generation are given a list of neighboring group nodes in the network, with whom they carry out a pBFT protocol with publicly verifiable secret sharing (PVSS). 

Within each group, a sharing threshold is set at $t = \lfloor \frac{N}{3} \rfloor + 1$ for group size $N$. Secret random seeds are generated by each participant, then encrypted shares are formed over that secret and distributed to other group members, along with cryptographic proofs on the shares.

For sharing threshold $t$, a random polynomial of order $t-1$ is generated $$p(x) = a_0 + a_1 x + ... + a_{t-1} x^{t-1}$$ with the secret value denoted by $a_0$.  Shares are constructed by computing the value of this polynomial for each member of the group, assigned successive ordinal values, $i = 1 ... N$. The resulting share values, $p(i)$, are then encrypted by multiplying the share value by the public key of each member, $E(share_i) = Zr(p(i)) P_i \in G2$, and proofs are generated by forming a point, $proof_i = Zr(p(i)) U \in G1$, for generator $U \in G1$. 

A vector of shares and a vector of proofs is generated, one element for each member of the group, and these vectors are then transmitted to each group member.
$$(E(share_1), E(share_2), ..., E(share_N))$$
$$ (proof_1, proof_2, ..., proof_N)$$

As with any BLS signature, each share is validated against its proof by checking that the pairings match:
$$ e(proof_i, P_i) = e(U, E(share_i))$$

Every member of the group can also verify that all shares from another group member were consistently generated from the same sharing polynomial. To do so, we treat the share vector as a codeword from a Reed-Solomon encoding\cite{scrape}, compute a random polynomial of order $N - t - 1$ and use that to compute a test vector from the dual-space of the original share generating polynomial:
$$f(x) = b_0 + b_1 x + ... + b_{N-t-1} x^{N-t-1}$$
$$c_{\perp} = (\lambda_1 f(1), \lambda_2 f(2), ... , \lambda_N f(N))$$
where weights $\lambda_i = \prod_{j \ne i} \frac{1}{i-j}$, for $ i,j = 1...N$.
Then the consistency of the encrypted shares is verified by checking that:
$$\sum_i {c_{\perp}}_i proof_i = G1(0)$$

This consistency check is absolutely certain for valid sharing vectors, and has an inconsequential probability of failing to detect an improper sharing set given as $\approx 1/q$, or about 1 chance in $2^{254}$. There is a greater likelihood of finding a hash collision in SHA3/256 than in seeing a failure to detect an inconsistent sharing vector.

After performing consistency checks on the sharing set from one group member, the share directed at one node can be decrypted with its secret key to produce a decrypted share, $$G2(share_i) = \frac{1}{s_i}E(share_i) \in G2$$ for secret key $s_i \in Zr$.

This decrypted share is then broadcast to all group members. Decrypted shares can be verified from the pairing relation:
$$ e(proof_i, V) = e(U, G2(share_i))$$

As soon as a sharing threshold number, $(n \ge t)$, of decrypted shares has been seen for any one sharing set, the secret randomness from that set can be discovered via Lagrange interpolation:
$$G2(random) = \sum_i G2(share_i) \prod_{j \ne i} \frac{i}{i-j}$$

Finally, after a supermajority of sharing sets has been decrypted, $(n \ge 2 \lfloor \frac{N}{3} \rfloor + 1)$, their randomness is combined as a simple sum in $G2$, and forwarded to all other groups.
$$G2(random_{grp}) = \sum_i G2(random_i)$$

Proof of group randomness comes from the sum of Lagrange interpolations of the individual proof sets.
Final randomness results from a supermajority sum of randomness obtained from each group, and its proof results from the sum of group proofs. 

So the use of pairing-based cryptography shows great benefits, not only in minimizing network traffic, and by making immediate commitments to portions along the way, and also from the fact that proofs are so easily generated as simple sums of existing proofs.

Timing tests show that this approach scales linearly with number of group participants, ranging from about 5 seconds for 32 group members, to about 7 minutes for 1024 group members, on an ordinary iMac with an Intel i7 processor. The timings are dominated by compute load, not network communications.

[ ... TBD: insert graph here ... ]

\chapter{Safe Hierarchical Keying}

In current blockchain designs which utilize simple Elliptic Curve cryptography, the possibility of producing subkeys from a master public key is presented. But that is wholly unsafe in the event that a decryption key is also generated for a derived public key. A simple bit of finite field arithmetic is all it takes to discover the original master private key.

With PBC we can safely generate both public and private keys without exposing our master private key. This is also known as Identity-Based Encryption (IBE). But unlike conventional presentations of IBE, we do not rely on a trusted third party for the generation of our keying. Rather, we view the master key holder as the only entity that should be entitled to generate new decryption keys. 

Anyone can generate new public keys at any time, based on previously known public keys.
But in order to obtain a decryption key for the new public keys, you must ask the primary secret key holder for a decryption key. Doing so puts the primary key holder at no risk for exposing his or her private key.

A new public key can be generated by asking for a subkey of a given public key, using an arbitrary identity value to identify that subkey. The new public key is computed as $$ P_{id} = Zr(H(id)) V + P$$ for identity $id$, generator $V \in G2$, public key $P \in G2$, and where $Zr(H(id)) \in Zr$ is the element of the field that corresponds to the hash of the supplied identity. 

You can use this public key to encrypt a message by making use of the hash of a pairing value as an XOR mask against a message $$E(msg) = msg \oplus H(g^r)$$ where $r = Zr(H(msg, id)) \in Zr$, and pairing element $g^r = e(r U, P_{id})$, for generator $U \in G1$. The message is transmitted as the triple $(E(msg), R, id)$, with $R = r P_{id} \in G2$.

In order to produce a decryption key for that new public key, the primary key holder computes $$s_{id} = \frac{1}{s + Zr(H(id))} U \in G1$$ for secret key $s \in Zr$. Producing a decryption key in $G1$ ensures, by difficulty of ECDLP, that our master private key remains safe against exposure.

Homomorphism allows us to see that the pairings 
$$e(s_{id} U, R) = e(\frac{1}{s + Zr(H(id))} U, r(Zr(H(id)) V + P)) = e(U, r V) = g^r$$ which allows us to recreate the XOR mask and decrypt to the original message $$msg = E(msg) \oplus H(g^r)$$ Verification of the message is done by computing $r = Zr(H(msg, id))$ and checking that $$r (Zr(H(id)) V + P) = R \in G2$$

In this form, a new private key cannot be used to sign messages in the same manner as for BLS signatures with the master private key. But it does furnish a way to encrypt and decrypt messages by using the hash of the pairing result. This technique has been dubbed SAKKE by its authors Sakai-Kasahara\cite{sakke}. We have extended SAKKE encryption to indefinite length by using successive SHA3 hashes on the pairing field result and an increasing index value.

\begin{thebibliography}{99}
%[ ... TBD: fill out the bibliography ... ]

\bibitem{scrape}Ignacio Casudo and Bernardo David, {\em{ SCRAPE: Scalable Randomness Attested by Public Entities}} 
\bibitem{thesis} Ben Lynn, PhD Thesis, {\em{On the Implementation of Pairing-Based Cryptosystems}}, June 2007
\bibitem{lib} Ben Lynn, PBC Library, https://crypto.stanford.edu/pbc/download.html
\bibitem{bls} Ben Lynn, {\em{BLS Signatures}}, https://crypto.stanford.edu/pbc/manual/ch02s01.html
\bibitem{sakke}Sakai-Kasahara, {\em{IETF RFC 6508 Sakai-Kasahara Key Encryption (SAKKE)}} \end{thebibliography} 

\end{document}
%=%=%=%=%=%=%=%=%=%=%=%=%=%=%=%=%=%=%=%=%=%=%=%=%=%=%=%
